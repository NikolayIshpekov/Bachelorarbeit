\chapter{Cross-Plattform-Entwicklung mit Xamarin}
  Xamarin ist 2011 infolge der Entlassung des fr�heren Teams zur Entwicklung von Mono durch Novell entstanden.
Xamarin ist ein unabh�ngiges Unternehmen, aber arbeitet eng mit Microsofts .NET-Team zusammen. Es
gibt eine Partnerschaft zwischen Microsoft und Xamarin, die dazu f�hrt, dass Visual-Studio-Entwickler 
durch Xamarins Produkte zur Entwicklung von Apps f�r iOS und Android unterst�tzt werden. Im Z�ge der 
 Partnerschaft k�nnen MSDN-Abonnenten au�erdem kostenlos auf den Kurs Xamarin University zugreifen, 
 der dabei helfen soll, innerhalb kurzer Zeit zum Experten der Xamarin-Tools zu werden \cite{MX}.
 Das Entwickeln in einer einheitlichen Programmiersprache ist nicht der einzige Vorteil von Xamarin.
 Xamarin.Forms ist ein m�chtiges Feature, mit dem man nicht nur die Anwendung in C\# entwickeln
 kann, sondern man auch den gr��ten Teil des Codes gemeinsam f�r alle drei Zielplattformen nutzen
 kann. Man kann mit Xamarin.Forms in einem Projekt gleichzeitig f�r iOS, Android und Windows Phone
 entwickeln und alle drei Produkte haben am Ende bis zu 70\% gemeinsamen Code und nur 30\%
 plattformspezifischen Code.

\section{Getting Started mit Xamarin}
Um Xamarin benutzen zu k�nnen, braucht man einen kostenpflichtigen Account. Es gibt verschiedene
Arten von Accounts, die mit unterschiedlichen Features verbunden sind.

In diesem Kapitel werden die drei wichtigsten Bibliotheken von Xamarin vorgestellt. 
Mit Xamarin.iOS / Xamarin.Android erstellte Programme stehen nativen Anwendungen in nichts nach und
k�nnen problemlos in App Store eingestellt werden. F�r erfahrene Entwickler ist der Einstieg in
Xamarin.iOS und Xamarin.Android ganz leicht, da es sich sehr der nativen App-Programmierung �hnelt,
aber im Gegensatz zu der �blichen Programmiersprachen (Objective-C und Java) wird bei Xamarin.iOS
und Xamarin.Android ausschlie�lich C\# benutzt.
Die plattform�bergreifende App-Entwicklung mit Xamarin geht noch weiter. Mit Xamarin.Forms lassen 
sich Apps f�r iOS, Android und Windows Phone mit gemeinsamen Code erstellen. Xamarin.Forms ist jedoch 
nicht f�r alle Typen von Anwendungen geeignet.
\section{Xamarin.iOS}
Mit Xamarin.iOS bietet Xamarin die M�glichkeit, native Apps f�r iOS zu entwickeln. Der Ahead-Of-Time
(AOT) Compiler kompiliert Xamarin.iOS Apps direkt zu nativen ARM Assemblercode, mit anderen Worten,
kann man mit Xamarin native iOS Apps erstellen. Mit Xamarin.iOS hat man quasi die iOS SDK von Apple
in C\# mit .NET Namenskonventionen. Xamarin bietet somit Zugriff auf alle iOS APIs. Dank eines 
automatischen Binding-Generators kann man existierenden Objective-C Code, Frameworks und
benutzerdefinierte Controls problemlos in einer Xamarin App benutzt werden.
