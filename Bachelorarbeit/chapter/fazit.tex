\chapter{Zusammenfassung und Ausblick}
\section{Zusammenfassung}
Die Arbeit beginnt mit der Einf�hrung in die reichhaltige Welt der mobilen Smartphones, wobei es
stark auf die st�ndig wachsende Rolle der Apps in der heutigen Welt akzentuiert wird. Es werden
Probleme erl�utert, die durch die Heterogenit�t der mobilen Betriebssysteme entstanden sind und mit
denen Entwickler oft zusammensto�en m�ssen. Es werden die aktuellsten mobilen Betriebssysteme und
typische Architekturmuster f�r die Entwicklung mobiler Anwendungen vorgestellt. Damit der Leser
einen Eindruck �ber die Unterschiede zwischen den plattformspezifischen Entwicklungsans�tze
bekommen kann, werden die Techniken f�r die Erstellung von iOS und Android Apps kurz
zusammengefasst. Aufbauend auf die Grundlagen der nativen App-Entwicklung, wird der Begriff
"`Cross-Plattform-Entwicklung"' gekl�rt und es werden die meist verbreiteten plattform�bergreifenden
Alternativen der nativen App Entwicklung vorgestellt. Anschlie�end wird die Wahl von Xamarin
begr�ndet und das Xamarin Framework wird n�her vorgestellt. \\Die zu
entwickelnde App bietet grundlegende Funktionen, um das Scannen eines Barcodes, die
dauerhafte Speicherung von Daten, sowie die Nutzung von Web Services zu erm�glichen.
Im Entwurfsteil werden Anforderungen an
die App festgelegt. Dar�ber hinaus werden wichtige Entscheidungen bez�glich Entwurfsmusters, Art des
Persistierens der Daten, sowie der Gestaltung der Navigation zwischen den Ansichten, getroffen. Die
Realisierung der Web Services f�r die Kommunikation der App mit dem Backend und die Benutzung von
externen Komponenten werden ebenso in dem Entwurfskapitel er�rtert. Anschlie�end wird einen Einblick
in die konkrete Implementierung und die Gestaltung und Ausf�hrung von UI-Tests gegeben.

\section{Ausblick}
Durch die gro�e Community und die langfristige Partnerschaft mit Microsoft scheint die Zukunft von
Xamarin gesichert zu sein. Es ist bemerkbar, dass das Team von Xamarin st�ndig an die
Weiterentwicklung des Frameworks arbeitet. Regelm��ig werden Aktualisierungen angeboten, vor allem
wenn eine neue iOS- oder Android-Version rauskommt, dauert es erfahrungsgem�� h�chstens ein bis zwei
Tage bis Xamarin das entsprechende Update anbietet. Allein die M�glichkeit native iOS-, Android- und
Windows Phones-Apps in einer einheitlichen, m�chtigen Programmiersprache, n�mlich C\# bringt
Entwicklern einen Mehrwert. 
Wenn man noch die Vorteile der gleichzeitigen Entwicklung von mehreren plattformspezifischen
Versionen einer App mit Xamarin.Forms dazu z�hlt und die M�glichkeiten, Apps in der Xamarin Test
Cloud auf Tausende von realen Ger�ten zu testen, f�llt einem nicht schwer zu schlie�en, dass
Cross-Plattform-Entwicklung mit Xamarin sich auf immer mehr Bef�rworter freuen wird.
