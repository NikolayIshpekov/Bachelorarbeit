\chapter{Zusammenfassung und Ausblick}
\section{Zusammenfassung}
Die Arbeit beginnt mit der Einf�hrung in die reichhaltige Welt der mobilen Smartphones, wobei es
stark auf die st�ndig wachsende Rolle der Apps in der heutigen Welt akzentuiert wird. Es werden
Probleme erl�utert, die durch die Heterogenit�t der mobilen Betriebssysteme entstanden sind und mit
denen Entwickler oft zusammensto�en m�ssen. Es werden die aktuellsten mobilen Betriebssysteme und
typische Architekturmuster f�r die Entwicklung mobiler Anwendungen vorgestellt. Damit der Leser
einen Eindruck �ber die Unterschiede zwischen den plattformspezifischen Entwicklungsans�tze
bekommen kann, werden die Techniken f�r die Erstellung von iOS und Android Apps kurz
zusammengefasst. Aufbauend auf die Grundlagen der nativen App-Entwicklung, wird der Begriff
"`Cross-Plattform-Entwicklung"' gekl�rt und es werden die meist verbreiteten plattform�bergreifenden
Alternativen der nativen App Entwicklung vorgestellt. Anschlie�end wird die Wahl von Xamarin
begr�ndet und das Xamarin Framework wird n�her vorgestellt. \\Die zu
entwickelnde App bietet grundlegende Funktionen, um das Scannen eines Barcodes, die
dauerhafte Speicherung von Daten, sowie die Nutzung von Web Services zu erm�glichen.
Im Entwurfsteil werden Anforderungen an
die App festgelegt. Dar�ber hinaus werden wichtige Entscheidungen bez�glich Entwurfsmusters, Art des
Persistierens der Daten, sowie der Gestaltung der Navigation zwischen den Ansichten, getroffen. Die
Realisierung der Web Services f�r die Kommunikation der App mit dem Backend und die Benutzung von
externen Komponenten werden ebenso in dem Entwurfskapitel er�rtert. Anschlie�end wird ein Einblick
in die konkrete Implementierung und die Gestaltung und Ausf�hrung von UI-Tests gegeben.\\Am Ende
dieser Arbeit wird die App und dadurch das plattform�bergreifende Appentwicklungsframework Xamarin
anhand ISO-Standards bewertet, dabei werden Feststellungen vorgestellt, die positiv oder auch
negativ ausgefallen sind.
\section{Ausblick}
Durch die gro�e Community und die langfristige Partnerschaft mit Microsoft scheint die Zukunft von
Xamarin gesichert zu sein. Es l�sst sich anmerken, dass das Team von Xamarin st�ndig an die
Weiterentwicklung des Frameworks arbeitet. Regelm��ig werden Aktualisierungen angeboten, so dass
Entwickler, die Xamarin benutzen, immer "`auf dem laufenden"' gehalten werden.\\Wie bei allen
relativ neuen Produkten auf dem Markt, gibt es noch einige verbesserungsbed�rftigen Stellen bei
Xamarin, wie bspw. der fehlende Designer zum Generieren von Xaml-Code bei Xamarin.Forms oder die
Tatsache, dass Mono nur eine Teilmenge von .NET ist und dadurch nicht alle .NET-Features
angewendet werden k�nnen.
\\Nichtsdestotrotz bringt alleine die M�glichkeit, native iOS-, Android- und Windows Phones-Apps in
einer einheitlichen, m�chtigen Programmiersprache, n�mlich C\#, einen gro�en Mehrwert f�r die
Entwickler. Wenn man noch die Vorteile der Entwicklung von mehreren plattformspezifischen Versionen
einer App, auf Basis eines gemeinsamen Codes mit Xamarin.Forms dazu z�hlt und die M�glichkeiten,
Apps in der Xamarin Test Cloud auf tausenden von realen Ger�ten zu testen, f�llt es nicht schwer zu
schlie�en, dass Cross-Plattform-Entwicklung mit Xamarin sich �ber immer mehr Bef�rworter freuen wird.\\Mit C\# und der Tatsache, dass eine Xamarin-App eine
reine native App ist, und nicht eine Web-App, die als native Applikation "`verpackt"' wurde,
liegt Xamarin weit vorne vor den konkurrierenden Cross-Plattform-Entwicklungsframeworks, wie bspw.
PhoneGap. \\Besonders f�r Business Apps, also Apps, bei denen
Funktionalit�t wichtiger als das Optische ist, scheint Xamarin das richtige Werkzeug zu sein.
\\Es l�sst sich zusammenfassen, dass Xamarin, vor allem mit der Benutzung einer
einheitlichen Programmiersprache, einem einheitlichen Architekturmuster (MVVM), einer einzelnen
Codebasis, aus der mehrere plattformspezifische Apps entstehen, und einer eigenen Test Cloud, nicht
nur anderen Cross-Plattform-Entwicklungsframeworks �berlegen ist, sondern sich auch zu einer realen
Gefahr f�r die nativen App-Entwicklungsans�tze in einer absehbarer Zukunft entwickeln kann, so dass
Cross-Plattform-Entwicklung sich als Standardl�sung behaupten kann, zumindest im
Bereich Entwicklung von Business Apps.
