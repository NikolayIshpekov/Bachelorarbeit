\chapter{Evaluierung inkl. Bewertungskriterien}

% \section{Anforderungen}
\section{Bewertungskriterien}
Es werden folgende Bewertungskriterien festgelegt:
\begin{itemize}
  \item Es soll eine plattform�bergreifende Programmiersprache benutzt werden.
  \item Gute Unterst�tzung durch IDE.
  \item Klare Trennung zwischen UI und Business Logik.
  \item Code-Sharing: Der Code-Sharing Anteil soll mehr als 50\% des Gesamtcodes sein.
  \item Gesch�ftslogik soll einmal in dem gemeinsamen Projekt implementiert werden und nicht in den
  plattformspezifischen Projekten.
  \item UI-Konzept - die von der Entwicklungsplattform bereitgestellten UI-Tools, sollen die
  Funktionalit�t einer nativ entwickelten App gew�hrleisten k�nnen.
  \item Persistierung (Datenbankanbindung) soll unterst�tzt werden.
  \item Es sollen m�glichst wenig Bugs w�hrend des Entwicklungsprozesses entstehen und deren
  Behebung soll nicht mit viel Aufwand verbunden sein.
  \item Die mit Xamarin entwickelte mobile Anwendung soll keine funktionale Unterschiede zu der
  Referenz Applikation aufweisen.
  \item Soll nicht deutlich langsamer als die Referenz Applikation sein.
\end{itemize}
\section{Evaluierung}
\subsection{Negative Eindr�cke}
\subsubsection{MVVM}
Aufgrund der nicht sehr ausf�hrlichen Informationen auf der offiziellen Seite von Xamarin ist es
keine einfache Aufgabe, sich an dem MVVM-Architekturmuster zu halten. 
Es ist zu beobachten, dass in vielen der Beispielen aus dem Netz, die ganze Funktionalit�t einer
Ansicht(Xaml-Datei) in der dazugeh�renden Code-Behind-Datei untergebracht wird, was die Idee
des MVVMs und die lose Kopplung zwischen View und Viewmodel zunichte macht. 
\\F�r einen Entwickler ohne Erfahrung mit MVVM-Pattern kann das sehr irref�hrend sein.
In den Beispielen, in denen das MVVM-Pattern sauber angewendet wird, werden oft einfache Szenarien
beschrieben, ohne komplizierte Interaktionen, wie z.B. Navigation zwischen den
Ansichten. 
\subsubsection{HTTPS}
Wie bereits erw�hnt wird das Protokol https benutzt, um eine sichere Kommunikation mit dem
Backend, in einer unsicheren Umgebung, zu gew�hrleisten. F�r die Nutzung von Web Services wird in
der Xamarin Comunity das NuGet Package \textit{Microsoft.Net.Http} empfohlen. Der in diesem Package
enthaltene Http-Client weist allerdings Schw�chen auf, wenn die Kommunikation wie im vorliegenden
Fall nicht �ber http sondern �ber https erfolgen soll. Diese Schw�chen sind kaum
dokumentiert worden und es kann lange dauern bis der Entwickler auf die Fehlerquelle kommt.
\subsubsection{Die Gr��e der IPA-Datei}
Die Datei ist deutlich gr��er als die IPA-Datei einer nativ entwickelten iOS-App.
\section{Grenzen der Cross-Plattform-Entwicklung}
Xamarin entwickelt sich mit raschem Tempo. Xamarin.iOS und Xamarin.Android bieten
.NET Entwicklern, die M�glichkeit Apps f�r iOS und Android zu entwickeln, die kaum von den
nativ entwickelten iOS- und Android-Apps zu unterscheiden sind. \\Interessant f�r die vorliegende
Arbeit ist der Ansatz von Xamarin.Forms. Wie im Kapitel 4.4.1.
bereits erl�utert, haben Entwickler f�nf verschiedenen Arten von Ansichten zur Auswahl, um  eine
Seite (Page) mit Xamarin.Forms zu erstellen. F�r die Implementierung der ScanApp werden lediglich
ContentPages, NavigationPages und eine MasterDetailPage f�r die Men�steuerung benutzt. \\Sollte es speziellere Anforderungen an der
Benutzeroberfl�che geben, m�ssen Entwickler sogenannte "`custom renderer"' benutzen, um auf native
plattformspezifische SDK-Features zugreifen zu k�nnen. Allerdings befinden sich die Xamarin
Renderer-APIs noch in einer Bearbeitungsphase (\cite{XamarinCustomRenderer}) und allem Anschein nach wird es noch
einige Zeit dauern bis Xamarin die M�chtigkeit der herk�mmlichen nativen App Entwicklung erreichen
kann.\\Es wurden oft die Vorteile der m�chtigen .NET Plattform erw�hnt. Allerdings muss beachtet
werden, dass Xamarin nicht alle diese Vorteile nutzen kann, weil Xamarin nicht direkt auf .NET
basiert, sondern auf das Open-Source-Projekt Mono. D.h. Xamarin-Developer haben nur
eine Teilmenge der Funktionalit�ten von .NET zur Verf�gung.


