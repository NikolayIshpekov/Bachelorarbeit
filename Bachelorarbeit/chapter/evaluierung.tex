\chapter{Evaluierung inkl. Bewertungskriterien}
Es werden folgende Bewertungskriterien festgelegt:
\begin{itemize}
  \item Es soll eine plattform�bergreifende Programmiersprache benutzt werden.
  \item Gute Unterst�tzung durch IDE.
  \item Klare Trennung zwischen UI und Business Logik.
  \item Code-Sharing: Der Code-Sharing Anteil soll mehr als 50\% des Gesamtcodes sein.
  \item Gesch�ftslogik soll einmal in dem gemeinsamen Projekt implementiert werden und nicht in den
  plattformspezifischen Projekten.
  \item UI-Konzept - die von der Entwicklungsplattform bereitgestellten UI-Tools, sollen die
  Funktionalit�t einer nativ entwickelten App gew�hrleisten k�nnen.
  \item Persistierung (Datenbankanbindung) soll unterst�tzt werden.
  \item Es sollen m�glichst wenig Bugs w�hrend des Entwicklungsprozesses entstehen und deren
  Behebung soll nicht mit viel Aufwand verbunden sein.
  \item Die mit Xamarin entwickelte mobile Anwendung soll keine funktionale Unterschiede zu der
  Referenz Applikation aufweisen.
  \item Soll nicht deutlich langsamer als die Referenz Applikation sein.
\end{itemize}
\section{Anforderungen}
\section{Bewertungskriterien}
\section{Grenzen der Cross-Plattform-Entwicklung}