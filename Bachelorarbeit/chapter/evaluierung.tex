\chapter{Evaluierung inkl. Bewertungskriterien}

% \section{Anforderungen}
\section{Bewertungskriterien}
Es werden folgende Bewertungskriterien festgelegt:
\begin{itemize}
  \item Es soll eine plattform�bergreifende Programmiersprache benutzt werden.
  \item Gute Unterst�tzung durch IDE.
  \item Klare Trennung zwischen UI und Business Logik.
  \item Code-Sharing: Der Code-Sharing Anteil soll mehr als 50\% des Gesamtcodes sein.
  \item Gesch�ftslogik soll einmal in dem gemeinsamen Projekt implementiert werden und nicht in den
  plattformspezifischen Projekten.
  \item UI-Konzept - die von der Entwicklungsplattform bereitgestellten UI-Tools, sollen die
  Funktionalit�t einer nativ entwickelten App gew�hrleisten k�nnen.
  \item Persistierung (Datenbankanbindung) soll unterst�tzt werden.
  \item Es sollen m�glichst wenig Bugs w�hrend des Entwicklungsprozesses entstehen und deren
  Behebung soll nicht mit viel Aufwand verbunden sein.
  \item Die mit Xamarin entwickelte mobile Anwendung soll keine funktionale Unterschiede zu der
  Referenz Applikation aufweisen.
  \item Soll nicht deutlich langsamer als die Referenz Applikation sein.
\end{itemize}
\section{Evaluierung}
\section{Grenzen der Cross-Plattform-Entwicklung}
Xamarin entwickelt sich mit raschem Tempo. Xamarin.iOS und Xamarin.Android bieten
.NET Entwicklern, die M�glichkeit Apps f�r iOS und Android zu entwickeln, die kaum von den
nativ entwickelten iOS- und Android-Apps zu unterscheiden sind. \\Interessant f�r die vorliegende
Arbeit ist der Ansatz von Xamarin.Forms. Wie im Kapitel 4.4.1.
bereits erl�utert, haben Entwickler f�nf verschiedene M�glichkeiten eine Seite (Page) mit Xamarin.Forms zu erstellen. F�r die Implementierung der N��ing ScanApp ist
diese Auswahl ausreichend. Es werden lediglich ContentPages, NavigationPages und eine
MasterDetailPage f�r die Men�steuerung benutzt. \\Sollte es speziellere Anforderungen an der
Benutzeroberfl�che geben, m�ssen Entwickler sogenannte "`custom renderer"' benutzen, um auf native
plattformspezifische SDK-Features zugreifen zu k�nnen. Allerdings befinden sich die Xamarin
Renderer-APIs noch in einer Bearbeitungsphase (\cite{XamarinCustomRenderer}) und allem Anschein nach wird es noch
einige Zeit dauern bis Xamarin die M�chtigkeit der herk�mmlichen nativen App Entwicklung erreichen
kann.\\Es wurden oft die Vorteile der m�chtigen .NET Plattform erw�hnt. Allerdings muss beachtet
werden, dass Xamarin nicht alle diese Vorteile nutzen kann, weil Xamarin nicht direkt auf .NET
basiert, sondern auf das Open-Source-Projekt Mono. D.h. Xamarin-Developer haben nur
eine Teilmenge der Funktionalit�ten von .NET zur Verf�gung.


