\chapter{Einleitung}
\section{Motivation / Problembeschreibung}
 Business Apps (mobile Softwareanwendungen) spielen eine immer gr��ere Rolle in der
heutigen Welt.
Eine der h�ufigsten Anforderung f�r eine Business Anwendung ist, dass die App auf mehreren Mobilplattformen l�uft.
Da auf den meisten Smartphones entweder Android oder iOS installiert ist, spielen diese beiden
Plattformen eine zentrale Rolle f�r die Entwicklung von mobilen Softwareanwendungen.
Ungl�cklicherweise sind Android und iOS inkompatibel zueinander. Das bedeutet, dass ein Entwickler,
der eine auf beiden Plattformen lauff�hige App entwickeln m�chte, gezwungen ist, mehrere
plattformspezifischen Sprachen zu beherrschen.
 Man braucht Java, wenn man f�r Android entwickeln m�chte und Objective-C f�r iOS Anwendungen.
 Auch Windows Phone wird als Zielplattform immer attraktiver und das bedeutet einen gro�en
 Aufwand, man muss quasi drei mal dieselbe Anwendung in drei verschiedenen Programmiersprachen
 implementieren.
 
 Ein Beispiel w�re eine mobile Anwendung, die auf iOS und Android laufen soll. 
 Wenn man 100 Stunden f�r die Entwicklung der iOS App und weitere 100 f�r die Android
  Version der Anwendung braucht, bedeutet das, dass man insgesamt 200 Arbeitsstunden einplanen
  muss. Aus gleichen Anforderungen resultieren zwei v�llig unabh�ngige und unterschiedliche
  L�sungen.
  \textbf{Cross-Plattform-Entwicklung} k�nnte hier Abhilfe schaffen. Die Idee ist, dass man anstatt
  200 Stunden, nur 130 - 150 Stunden in die Entwicklung der App stecken muss.Also Softwarehersteller
  k�nnen somit die Entwicklungskosten deutlich reduzieren. Dar�ber hinaus hat sich der
  Smartphonemarkt als unbest�ndig erwiesen und plattformgebundene App Entwicklung ist demzufolge ein riskantes
  Unterfangen. 
  
  Cross-Plattform-Technologien scheinen die Zukunft der App Entwicklung zu sein. Allerdings
  gestaltet sich die Auswahl einer geeigneten plattform�bergreifenden Technologie als eine gar nicht
  so einfache Aufgabe. Diese Technologien sind relativ neu und es fehlt immer noch einen umfassenden
  �berblick �ber die Funktions- und Leistungsf�higkeit eines solchen Produktes. Die vorliegende
  Arbeit befasst sich mit Xamarin. Es wird eine praxisnahe Evaluation, anhand eines Fallbeispiels,
  durchgef�hrt. Dabei wird dargelegt, wie Xamarin als Cross-Plattform verwendet werden kann, um
  native Anwendungen f�r Android und iOS zu erstellen und welche Vorteile, Nachteile und
  Beschr�nkungen entstehen.
 Lohnt es sich f�r Softwarehersteller in Xamarin zu investieren? Ist Xamarin f�r
 Business Apps geeignet und welche Vor- und Nachteile bringt Xamarin mit? Im Laufe dieser
 Bachelorarbeit wird eine App nachgebaut, die schon f�r iOS und Android entwickelt wurde. Dabei ist
 interessant herauszufinden, an welchen Stellen es Probleme entstehen k�nnen und ob es �berhaupt m�glich ist,
  die App mit Xamarin zu entwickeln, so dass es keine Leistungsverluste gibt. Die ben�tigten
  Installationsschritte, sowie Hardware- und Softwarevoraussetzungen werden erl�utert. 
\section{L�sungsidee}


\section{Aufgaben und Ziele / Bewertungskriterien}
Ziel der Arbeit - eine Cross-Plattform-Entwicklungsumgebung zu evaluieren f�r die Entwicklung von
mobilen Enterprise-Anwendungen f�r heterogene Plattformen. (Anforderungen und Bewertungskriterien
stichpunktartig festlegen)


