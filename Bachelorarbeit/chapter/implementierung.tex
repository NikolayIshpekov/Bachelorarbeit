\section{Implementierung und Test}
\subsection{Implementierung}
\subsection{Test}
Ein wichtiger Schritt im Entwicklungszyklus einer App ist das Testen. 
Ein typischer Weg, eine App zu testen ist die zu starten und zu benutzen. Im besten Fall tut die App
genau das was sie tun soll - funktioniert korrekt und st�rzt nicht ab. Erfahrungsgem�� ist das so
gut wie nie der Fall. Diese Art von Testen, indem man die App auf einem realen Ger�t nutzt, wird als UI
Acceptance Testing bezeichnet.
Die zur Verf�gung gestellten Simulatoren erleichtern wesentlich den Testprozess, allerdings um
sicher zu gehen, dass eine App wirklich fehlerfrei funktioniert, kommt man an Tests auf reale Ger�te nicht
herum. Da es vor allem bei Android eine gro�e Vielfalt an Ger�ten gibt, kann das sehr m�hsam sein.
Softwarehersteller m�ssen nicht selten Apps auf dutzende sogar hunderte Ger�te installieren und
testen.\\Die meisten Developer verzichten auf systematisches Testen, weil die verf�gbaren Tools und
Dienste zu kompliziert und schwer zu benutzen sind.
Alle Xamarin-Plattform-Abonnements beinhalten 60 Xamarin Test Cloud Ger�te-Minuten pro Monat. D.h.
jeder Entwickler, der einen Xamarin Account hat, kann die Dienste der Xamarin Test Cloud in Anspruch nehmen.
Xamarin verf�gt �ber 1600 reale iOS und Android Smartphones und Tabletts. Mit Xamarin Test Cloud
kann man leicht visuelle Inkonsistenzen feststellen, indem man die Ergebnisse auf dutzende Ger�ten
vergleicht. Es werden Screenshots abgebildet. Der Service bietet auch Videoaufnahme von Tests.
Eine Ger�te-Minute wird konsumiert, wenn der Test auf einem Ger�t ausgef�hrt wird, wobei es keine
Rolle spielt ob die Tests parallel auf mehrere Ger�te oder nacheinander ausgef�hrt werden.

