\chapter{Cross-Plattform-Entwicklung von mobilen Anwendungen und Stand der Technik}
\section{Was ist Cross-Plattform-Entwicklung - Begriffskl�rung}
\section{Vorstellung und Bewertung der aktuellen Cross-Plattformen}

\cite{10BestCPMDT}
\subsection{PhoneGap}
PhoneGap ist 
PhoneGap ist wahrscheinlich der bekannteste Cross-Plattform-Brand. Das Tool ist mittlerweile
Eigentum von Adobe und basiert auf der open source Apache Cordova Projekt. Besonders vorteilhaft
ist, dass es komplett kostenlos ist. Adobe hat eine Enterprise Version von PhoneGap annonciert, bei
der Features via Adobe's Marketing Cloud eingebunden sein werden.
\subsection{Web Apps}
Webbasierte Applikationen werden �ber das Internet angeboten und umgehen so die Restriktionen und
Rahmenbedingungen plattformspezifischer Marktpl�tze. Webanwendungen k�nnen auf jedem beliebigen
mobilen Endger�t ausgef�hrt werden, solange das Endger�t �ber ein Internetbrowser verf�gt. Web Apps
werden auf der Grundlage von HTML, CSS oder JavaScript entwickelt. Die ben�tigen keine Installation
und werden �ber das Internet zur Verf�gung gestellt. Aufgrund fehlender Schnittstellen weisen
Webbasierte Anwendungen einen eingeschr�nkten Zugriff auf die Hardwareressourcen des jeweiligen
Endger�tes auf. Sogenannte hybride Anwendungen basieren zum einen auf einem nativen Kern, der die
Verbindung zu den Hardwareressourcen erm�glicht, und zum anderen auf webbasierten,
plattform�bergreifenden Funktionen. Obwohl spezielle Frameworks die Umwandlung von
plattformunabh�ngigen Anwendungen in native Apps erm�glichen, nutzen viele Hersteller, wie Apple
oder Google, ihre Marktdominanz aus, um hybride Technologien und Frameworks zu verhindern. Weitere
Nachteile von hybride Apps sind die mangelnde Sicherheit und die fehlende Offenheit
\cite[106]{App4U}.
\subsection{Appcelerator Titanium Mobile}
\subsection{Xamarin}
\section{Plattformauswahl}